%%%%%%% Chapter 1 %%%%%%%


\chapter{绪论}

此处格式已按模板设定,作者只需选择段落区域,输入替换之。模版中所有说明性文字用于注释格式与内容的要求,撰写论文时请删除。模版中,图表、公式、参考文献等都已给出范例,撰写论文时请删除。

本模版已包含符合章节设置的“多级别列表”,只需在相应位置替换标题文字即可。如需增加章节,建议先使用格式刷,再调整编号。

\section{毕业设计(论文)类型及基本要求}

\subsection{设计类}

学生必须独立完成一定数量的设计图纸,撰写一篇15000字以上的设计说明书,工程设计类图纸折合成零号图纸不能少于三张。

\subsection{论文类}

学生必须独立完成一项研究或实验,撰写一篇15000字以上的论文。基础理论类的研究型论文的正文文字一般不少于10000字,要求内容充实,论据充分可靠,论证有力,主题明确。

\subsection{软件类}

学生必须独立完成一个软件或较大软件中的一个模块设计,撰写一篇15000字以上的设计说明书。

\section{毕业设计(论文)资料组成及存档要求}

毕业设计(论文)资料包括:任务书、开题报告、毕业设计(论文)、外文资料、中文译文、过程指导记录表、中期检查记录表、指导教师评阅书、评阅教师评阅书、答辩记录书,以及图纸、实验报告、计算程序等。

毕业设计(论文)的纸质版资料由各院、系自行安排保存,电子版资料在天津大学本科生毕业设计(论文)管理系统中保存,要保证两类资料的一致性。

毕业设计(论文)及图纸、实验报告、计算程序等装订成册(图纸数量过多可单独装订成册),附件资料按顺序装订成册(资料顺序为:附件封面、目录、任务书、开题报告、外文资料、中文译文、过程指导记录表、中期检查记录表、指导教师评阅书、评阅教师评阅书、答辩记录书),两册一并存档,两册封面均采用天津大学本科生毕业设计(论文)统一封面。
